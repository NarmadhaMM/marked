\batchmode
\makeatletter
\def\input@path{{C:/Users/JLaake/git/marked/manuscript//}}
\makeatother
\documentclass[12pt]{article}
\usepackage[latin9]{inputenc}
\usepackage{geometry}
\geometry{verbose}
\usepackage{amsmath}
\usepackage{amssymb}
\usepackage[authoryear]{natbib}

\makeatletter

%%%%%%%%%%%%%%%%%%%%%%%%%%%%%% LyX specific LaTeX commands.
%% Because html converters don't know tabularnewline
\providecommand{\tabularnewline}{\\}

%%%%%%%%%%%%%%%%%%%%%%%%%%%%%% Textclass specific LaTeX commands.

%%%%%%%%%%%%%%%%%%%%%%%%%%%%%% User specified LaTeX commands.

\usepackage{amsthm}\usepackage{epsfig}\usepackage{psfrag}\usepackage{lineno}

\bibliographystyle{apalike}

%\setlength{\evensidemargin}{0in} \setlength{\oddsidemargin}{0in}
%\setlength{\topmargin}{0.0in} \setlength{\textwidth}{6.5in}
%\setlength{\textheight}{9in} \setlength{\topskip}{0in}
%\setlength{\headheight}{0in} \setlength{\headsep}{0in}

\makeatother

\usepackage{Sweave}
\begin{document}
%% TITLE %%%%%%%%%%%%%%%%%%%%%%%%%%%%%%%%%%%%%%%%%%%%%%%%%
\global\long\def\baselinestretch{1.8}


\begin{center}
\texttt{\LARGE marked}{\LARGE : An R package for maximum-likelihood
and MCMC analysis of capture-recapture data}\vspace{0.5in}

\par\end{center}

\begin{center}
{\large Devin S. Johnson$^{1}$, Jeff L. Laake, and Paul B. Conn}
\par\end{center}{\large \par}

\begin{center}
\hrulefill{}
\par\end{center}

\begin{center}
\global\long\def\baselinestretch{1.25}
{\large National Marine Mammal Laboratory, Alaska Fisheries Science
Center}\\
{\large{} NOAA National Marine Fisheries Service,}\\
{\large{} Seattle, Washington, U.S.A.}\\
{\large{} $^{1}${\em Email:} devin.johnson@noaa.gov}
\par\end{center}{\large \par}

\begin{center}
{\large \hrulefill{}}
\par\end{center}{\large \par}

\begin{center}
\textsc{Running Head}: Running header \bigskip{}

\par\end{center}

\begin{center}
July 13, 2012 
\par\end{center}

\clearpage{}

%% ABSTRACT %%%%%%%%%%%%%%%%%%%%%%%%%%%%%%%%%%%%%%%%%%%%%%%%%%%%%


%{\sc Summary.}
\textbf{Summary}
\begin{enumerate}
\item 
\item 
\item 
\end{enumerate}
\noindent \hrulefill{}

\noindent \textsc{Key words} Words here

\clearpage{}

\linenumbers

\global\long\def\baselinestretch{1.8}


Currently the most comprehensive software for analysis of capture-recapture
data is MARK (\citealt{White1999}). MARK is a FORTRAN program for
fitting capture-recapture models that are manually constructed with
a graphical user interface. RMark (\citealt{Laake2008}) is an R package
that constructs models for MARK with user-specified formulas to replace
the manual model creation. With RMark and MARK most currently available
capture-recapture models can be fitted and manipulated in R. Yet,
additional R packages have been made available including Rcapture
(\citealt{Baillargeon2007}), mra (\citealt{McDonald2005}), secr
(\citealt{Borchers2008}), BTSPAS (\citealt{Schwarz2009}), SPACECAP
(\citealt{Royle2009}), and BaSTA (\citealt{Colchero2012}). Rcapture
fits closed and open models in a log-linear framework. The mra package
fits Cormack-Jolly-Seber (CJS) and the Huggins closed model with a
``regression approach'' to model specification. The secr and SPACECAP
packages provide spatially explicit modelling of closed capture-recapture
data and BTSPAS fits time-stratified Petersen models in a Bayesian
framework. BaSTA estimates survival with covariates from capture-recapture/recovery
data in a Bayesian framework when many individuals are of unknown
age. Each package is designed for a unique niche or structure. We
believe these alternative packages in R are useful because they expand
the analyst's toolbox and the code is open source which enables the
knowledgeable user to understand fully what the software is doing.
Also, this independent innovation provides testing for existing software
and potential gains in developer knowledge to improve existing software.

We developed an R package we named ``marked'' based on our focus
of analysis with marked animals in contrast to the R package unmarked
(\citealt{Fiske2011}) that focuses on observations of unmarked animals.
The original impetus for the code was to implement the CJS model using
the hierarchical likelihood construction described by \citet{Pledger2003}
with the added capability of variable time intervals for different
groups of animals or when initial capture/release is done between
recapture occasions. The latter are not easily accomplished in MARK.
Also, we wanted to improve on execution times with RMark/MARK for
analysis of our own large data sets with many time-varying individual
(animal-specific) covariates using the data structure idea in the
mra package (\citealt{McDonald2005}). Subsequently, we implemented
the Jolly-Seber model with the \citet{Schwarz1996} POPAN parameter
structure with the hierarchical likelihood construction idea extended
to the entry of animals into the population. We also added a Bayesian
Markov Chain Monte Carlo (MCMC) implementation of the CJS model based
on the approach used by \citet{ALBERT:1993fk} for analyzing binary
data with a probit regression model. 


\section*{Data structure}

The likelihood for CJS/JS is a product multinomial and the cell probabilities
are functions of the parameters. We will focus on CJS which has parameters
$p$ (capture/recapture probability) and $\phi$ (apparent survival).
If the parameters depend only on time, cohort or group (e.g., sex)
variables, the MARK approach using parameter index matrices (PIMs)
is very efficient because it assigns a potentially unique index for
each parameter type and position. The design matrix for the parameters
has a row for each unique parameter and the index is the row number.
The PIM approach minimizes the work of manual model creation by reducing
the set of potential parameters. As a very simple example, we will
consider a CJS model for a single group with k=4 occasions. For this
structure the PIMS are triangular with a row for each cohort released
at each occasion (time) and the column representing the occasions
(times) following the release. Using an unique index for each potential
parameter, the PIMS would be:

\begin{tabular}{ccccccccc}
 &
\multicolumn{3}{c}{} &
 &
 &
\multicolumn{3}{c}{}\tabularnewline
\hline 
 &
\multicolumn{3}{c}{$\phi$ } &
 &
 &
\multicolumn{3}{c}{$p$}\tabularnewline
\hline 
 &
\multicolumn{3}{c}{Time} &
 &
 &
\multicolumn{3}{c}{Time}\tabularnewline
\cline{2-5} \cline{7-9} 
Cohort &
2 &
3 &
4 &
 &
Cohort &
2 &
3 &
4\tabularnewline
\cline{2-4} \cline{7-9} 
1 &
1 &
2 &
3 &
 &
1 &
7 &
8 &
9\tabularnewline
2 &
 &
4 &
5 &
 &
2 &
 &
10 &
11\tabularnewline
3 &
 &
 &
6 &
 &
3 &
 &
 &
12\tabularnewline
\hline 
 &
 &
 &
 &
 &
 &
 &
 &
\tabularnewline
\end{tabular}

\noindent The index 5 represents survival of the second release cohort
between occasions 3 and 4 and index 8 represents recapture probability
of the first release cohort on the third occasion. Some constraints
on parameters (e.g., constant $\phi$) can be constructed by modifying
the PIMS but more generally models are constructed with a design matrix
which is a set of linear constraints. For our example, there would
be 12 rows in the design matrix and each column in the design matrix
($X$) is one of the model parameters $\beta$ which for a logit link
are related to real parameters $\theta$ (e.g., $\phi$ and $p$ in
CJS) by:

\noindent 
\begin{equation}
\theta=\frac{1}{1+\exp(-X\beta)}\label{eq:link}
\end{equation}
RMark creates design matrices for MARK by applying formula to ``design
data'' which are data about the parameters like cohort, time and
group and any related variables. However, having a single $X$ becomes
problematic with individual covariates (variables that differ for
animals in the same group/cohort). In MARK, these individual covariates
are handled by entering the name of the individual covariate in the
design matrix. When the real parmeters are computed, the name of the
covariate is replaced with the value for an animal. If there are only
a few individual covariates and small number of aninals, the replacement
and computation is not too costly but when the individual covariates
differ in time, there is a different covariate name for each time
and the computation cost can become quite extreme for large models. 

Even if computation time is not limiting, we agree with \citet{McDonald2005}
that the PIM structure used in MARK can be quite confusing to the
novice user. While PIMS are useful for manual model creation, they
become an unnecessary nuisance when models and design matrices are
created by formula. The solution of \citet{McDonald2005} (implemented
in mra) is to create covariate matrices with a row for each animal
and a column for each occasion and each covariate matrix is a prediction
variable as in regression. This is equivalent to creating a real parameter
for each animal for each occasion regardless of the release cohort.

We use an alternative approach that we believe is simpler for the
user. From the raw data and specified model structure, the code creates
a list containing a dataframe for each parameter in the model (e.g.,
$\phi$ and $p$). Each dataframe contains a record for each animal-occasion
and the columns are the covariates. For $\phi$ and $p$ in the CJS
model, there are $n(k-1)$ rows for $n$ animals and $k-1$ occasions
that are modelled. For the JS model, there are $n(k-1)$ records for
survival and probability of entry parameters and $nk$ for $p$ because
the initial capture event is modelled. We create a separate dataframe
for each parameter to enable different data (e.g., time values for
$\phi$ and $p$) and different structures and number of occasions
(e.g., $\phi$ and $p$ in JS). With model.matrix, a formula is applied
to the dataframe to create the design matrix for each parameter (e.g,
$X_{\phi}$,$X_{p}$). They are equivalent to a portion of the design
matrix in MARK for each animal which are then stacked on top of one
another to make a matrix for all animals. With maximum likelihood
estimation (MLE), for each type of parameter, equation \ref{eq:link}
(or similar inverse link function) is applied and the resulting vector
is converted to a matrix with $n$ rows and a column for each required
occasion ($k-1$ or $k$). For Bayesian MCMC inference, only$X\beta$
are needed for updating.

The parameter-specific animal-occasion dataframes are automatically
created from the typical user's dataframe containing a single record
per animal containing the capture history and covariates. For the
particular model structure, fields like cohort, age, and time are
generated for each record. Other static and time-varying covariates
specified in the user's data are added. Static variables (e.g., sex)
are repeated for each occasion for the animal. Time-varying variables
(e.g., age) vary across rows for each animal. They are specified in
the data using a naming convention of ``vt'' where ``v'' is the
covariate name and ``t'' is a numeric value for the time (occasion)
(e.g., td19). They are collapsed in the animal-occasion dataframes
to a single column with the name identified by ``v'' and each record
contains the appropriate value for the occasion. All of this is transparent
to the user who only needs to specify their dataframe, the type of
model (e.g., cjs), the variables that should be treated as time-varying,
and the formula for each parameter.

With this structure, the design matrices can become quite large and
available memory could become limiting. For MLE analyis, we use the
following to reduce the required memory:
\begin{enumerate}
\item We reduce the data to $n^{*}\leq n$ individuals by aggregating records
with identical data and using the frequencies ($f_{1},...,f_{n^{*}};n=\sum_{i=1\:}^{n^{*}}f_{i}$)
in the likelihood. 
\item For $X$ we use sparse matrices which only store non-zero elements
and construct $X$ incrementally with a user-specified size of data
chunks.
\item We retain only the rows of $X$ which are unique for the design and
including any fixed parameters which can be animal-specific. 
\end{enumerate}
For Bayesian MCMC inference, we cannot aggregate records but we reduce
the required memory by:
\begin{enumerate}
\item Eliminating the unused animal-occasion data for occasions prior to
and including the release occasion for the animal, and
\item Storing only the unique values of the real parameters which are unique
rows of $X_{\phi}$and $X_{p}$.
\end{enumerate}

\section*{Maximum-likelihood inference}

Although it was not explicitly stated, the CJS likelihood was developed
hierarchically by \citet{Pledger2003}. Considering a single release
cohort for simplicity, the probability of capture history ($\omega$
a sequence of 0 and 1 values) was computed by multiplying the conditional
probability ($Pr(\omega|d)$) for a given departure occasion (i.e.,
death or end of study) and the probability of departing on that occasion
($Pr(d)$) and summing across all possible departure occasions ($Pr(\omega)=\sum_{d}Pr(\omega|d)Pr(d)$).
Viewed in this way, it is possible to construct the likekihood values
for all of the observations with matrices. 

Let $\phi$,$M$ $p$ be $n\times k$ matrices where$\phi_{ij}$ is
the survival probability for animal $i$ from occasion $j-1$ to $j$
($\phi_{i1}=1$ ), $m_{ij}$ is the probability of dying in the interval
$j$ to $j+1$ ($m_{ik}=1)$ and $p_{ij}$ is the capture probability
for animal $i$ on occasion $j$ ($p_{i1}=0$). Let $C$ be an $n\times k$
matrix of the capture history values and $f_{i}$and $l_{i}$ are
the first and last occasions the $i^{th}$ animal was seen. Derived
from those values are $n\times k$ matrices $L$, $F$ and $F^{+}$
which contain 0 except that $L_{ij}=1\; j\geq l_{i}$,$F_{ij}=1\; j\geq f_{i}$
and $F_{ij}^{+}=1\; j>f_{i}$. The likelihood calculation has the
following steps where the matrix mulitplication is elementwise and
$1$ and represents an $n\times k$ matrix where each element is 1:
\begin{enumerate}
\item Construct the $n\times k$ matrix $\phi'=1-F^{+}+\phi F^{+}$which
is the $\phi$ matrix modified such that $\phi_{ij}=1$ for $j\leq f_{i}$.
\item Construct the $n\times k$ matrix $\phi*$from $\phi'$ where $\phi_{ij}*=\prod_{i=1}^{j}\phi_{ij}'$.
\item Construct the $n\times k$ matrix $p'=1-F^{+}+F^{+}(Cp+(1-C)(1-p))$.
\item Construct the $n\times k$ matrix $p*$from $p'$ where $p_{ij}*=\prod_{i=1}^{j}p_{ij}'$.
\item Compute $n\times1$ vector of probabilities of observed capture histories
$Pr(\omega_{i})\; i=1,...,n$ which are the sums of the rows of $LM\phi^{*}p^{*}$.
\item Compute the log-likelihood $\sum_{i=1}^{n}\ln(Pr(\omega_{i}))$
\end{enumerate}
The R code translates from the mathematics quite literally as: 

\begin{Schunk}
\begin{Sinput}
>  Phiprime=1-Fplus + Phi*Fplus
>  Phistar=t(apply(Phiprime,1,cumprod))
>  pprime=(1-Fplus)+Fplus*(C*p+(1-C)*(1-p))
>  pstar=t(apply(pprime,1,cumprod))
>  pomega=rowSums(L*M*Phistar*pstar)
>  lnl=sum(log(pomega))
\end{Sinput}
\end{Schunk}

\noindent While this code is simple it is faster if the apply is replaced
with a loop because there are far more rows than occasions or replaced
with compiled code which we did with a FORTRAN subroutine.

The Jolly-Seber likelihood can be simplified by defining $\omega'$
to be the portion of $\omega$ up through the (e.g., $\omega=(001010)$
$\omega'=(001)$). The likelihood can be partitioned into 3 components:
1) CJS likelihood for $\omega$ treating the first ``1'' as a release,
2) a likelihood component for $\omega'$; entry and first observation,
3) a component for those that entered before each occasion but were
never seen. We define an additional $n\times k$ matrix $\pi$ which
are the entry probabilities into the population (specified as $\beta$
by \citet{Schwarz1996}) with the obvious constraint that $\sum_{j=1}^{k}\pi_{ij}=1$
and $N_{g}\: g=1,...,G$ is the abundance of animals in each of the
defined $G$ groups (e.g., male/female) that were in the population
at some time. $N_{g}=n_{g}+f_{g}^{0}$ where $n_{g}$ is the number
observed in the group and $f_{g}^{0}$ are the estimated number of
animals in the group that were never seen.

The same hierarchical approach used for CJS can be used for the second
component. Construct the capture history probability for a given entry
time and then sum over all possible entry times. . The second component
is constructed with the following steps:
\begin{enumerate}
\item Construct the $n\times k$ matrix $E=(1-p)\phi(1-F)+F$,
\item Construct the $n\times k$ matrix $E*$where $E_{ij}*=\prod_{l=j}^{k}E_{il}$,
\item Compute $n\times1$ vector of probabilities $Pr(\omega')$ which are
the sums of the rows of $E*(1-F^{+})\pi$ multiplied by the vector
$p_{if_{i}}\: i=1,...,n$, 
\item Compute the log-likelihood $\sum_{i=1}^{n}\ln(Pr(\omega_{i}'))$.
\end{enumerate}
For the third likelihood component for missed animals, we constructed
$G\times k$ dummy capture histories of all 0's except for a ``1''
at the occasion the animals entered. From the CJS portion of the code,
we obtained $p_{gj}^{0}$ the probability that an animal released
in group $g$ on occasion $j$ would never be sighted. The final log-likelihood
component is:
\[
\sum_{g=1}^{G}f_{g}^{0}\ln\left[\sum_{j=1}^{k}\pi_{gj}(1-p_{gj})p_{gj}^{0}\right]+\ln(N_{g}!)+\ln(f_{g}^{0}!)
\]


The JS log-likelihood is the total of the 3 components plus $\sum_{g=1}^{G}\sum_{j=1}^{k}\ln(n_{gj}!)$
which do not depend on the parameters but is added after optimization
to be consistent with the output from MARK.


\section*{Bayesian MCMC inference}

To implement a fairly \textquotedbl{}black box\textquotedbl{} MCMC
algorithm that is accessible to end users we took the approach used
by \citet{ALBERT:1993fk} for analyzing binary data with a probit
regression model.


\section*{Getting initial parameter values}

Initial values for parameters can either be provided from the results
of a previously fitted similar model or they are computed using general
linear models (GLM) that provide approximations. Using the underlying
idea in \citet{Manly1968} we compute initial estimates for capture-probability
$p$ for occasions 2 to $k-1$ using a binomial GLM with the formula
for $p$ which is fitted to a sequence of bernoulli random variables
that are a subset of the capture history values $y_{ij}$ $i=1,...,n$
and $j=f_{i}+1,...,l_{i}-1$ where $f_{i}$ and $l_{i}$ are the first
and last occasions the $i^{th}$ animal was seen. A similar but more
ad-hoc idea is used for $\phi$. We know the animal is alive between
$f_{i}$ and $l_{i}$ and assume that the animal dies at occasion
$l_{i}+1\leq k$. We use a binomial GLM with the formula for $\phi$
fitted to the $y_{ij}^{*}$ $i=1,...,n$ and $j=f_{i}+1,...,l_{i}+1\leq k$
where $y_{ij}^{*}$=1 for $j=f_{i}+1,...,l_{i}$ and $y_{ij}^{*}$=0
for $j=l_{i}+1\leq k$. A logit link is used for MLE and a probit
link for MCMC. Any non-estimable or missing $\beta$ are set to 0. 


\section*{Acknowledgments}

\bibliography{0C__Users_JLaake_git_marked_manuscript_markedWriteUpBib}

\end{document}
